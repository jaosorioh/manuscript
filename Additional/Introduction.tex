% \chapter*{Abstract}
\chapter*{Introduction}
\chaptermark{Introduction}
\label{chap:Introduction}
\addcontentsline{toc}{chapter}{Introduction}
%Insert your abstract below the line
%-------------------------------------------
The existence of nearly all fundamental particles in the universe has been successfully predicted and explained by the Standard Model (SM) \cite{stiller2016full}. The Large Hadron Collider (LHC) has played an important role in proving several of these predictions \cite{baron2018desarrollo}. With the high energy provided by the LHC, states of matter such as the Quark-Gluon Plasma (QGP) can be achieved. The study of heavy-quark production is important in order to understand some of the QGP properties \cite{aziz2021z}. The $B^0_s$ meson is of special interest in this matter since it provides a clean observation of the charge parity (CP) violation, which can be used to test how good the SM model is and obtain some hints of the physics beyond it \cite{greevenanalysis}.

The primary goal of this dissertation is to measure the differential cross-section of the $B^0_s$ meson using the decay chain $B^0_s \to J\psi \phi \to \mu^{+}\mu^{-} K^{+}K^{-}$ with data corresponding to 2016 p-Pb collisions at the CMS at  $\sqrt{s_{NN}} = 8.16$ TeV and $\mathcal{L} = 179.1$ nb$^{-1}$. This same quantity has been measured previously by \cite{khachatryan2016study}, where they considered $\sqrt{s_{NN}} = 5.02$ TeV and $\mathcal{L} = 34.6$ nb$^{-1}$. \cite{chatrchyan2011measurement} measured it in pp collisions at $\sqrt{s_{NN}} = 7$ TeV with $\mathcal{L} = 40$ fb$^{-1}$ and \cite{canelli2019measurement} in pp and PbPb collisions at $\sqrt{s_{NN}} = 5.02$ TeV with $\mathcal{L_{\text{pp}}} = 28$ pb$^{-1}$ and $\mathcal{L_{\text{PbPb}}} = 28$ pb$^{-1}$ + 351 $\mu b^{-1}$ respectively. We present here an updated value in comparison to previous measurements.

The manuscript is divided into chapters that explain the conceptual, experimental, and methodological concepts that underpin the research. The standard model, quantum chromodynamics, and the $B^0_s$ meson are briefly described in chapter \ref{chap:chapter_1}. The experimental facilities of the Large Hadron Collider (LHC) are given in chapter \ref{chap:chapter_2}, as well as a full explanation of the Compact Muon Selenoid (CMS) detector and its most important components. The procedure of generating Monte Carlo samples and the selection criteria for the meson reconstruction are described in chapter \ref{chap:Chapter_3}. The statistical methods used to measure the invariant mass of $B^0_s$ are presented in detail in Chapter \ref{chap:Chapter_4}, along with the spectra results. The acceptance, efficiency, and total efficiency of reconstruction, as well as the differential cross-section in terms of transverse momentum, are determined in Chapter \ref{chap:Chapter_5}. The statistical and systematic uncertainties and the methods for calculating them are discussed in the same chapter. Finally, the conclusions of this work are presented in chapter \ref{chap:Chapter_6}. 