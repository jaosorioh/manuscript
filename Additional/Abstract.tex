% \chapter*{Abstract}
\chapter*{Abstract}
\chaptermark{Abstract}
\label{chap:Abstract}
\addcontentsline{toc}{chapter}{Abstract}
%Insert your abstract below the line
%-------------------------------------------
In this thesis, the $B^0_s$ meson spectrum has been measured through the decay channel $B^0_s \to J/\psi \phi \to \mu^{+}\mu^{-} K^{+}K^{-}$ using the data from p-Pb collisions collected during 2016 by the CMS detector at LHC with center of mass energy $\sqrt{s_{NN}} = 8.16$ TeV and total integrated luminosity 179.1 nb$^{-1}$. Monte Carlo simulated samples with similar conditions to that of the collision have been considered as well. The measurement was achieved by means of a probability density function for the background and signal events and the implementation of the extended maximum likelihood method with the \verb|ROOTFIT| package. 

Later on, the differential cross-section with respect to the transverse momentum $p_T$, as well as the sources of systematic and statistical uncertainty, were calculated. A comparison was made with the differential cross-section from FONLL. The following are the results obtained:

$$M(B^0_s) = 5367.02 \pm 0.86 MeV$$
$$\left(\frac{d \sigma(B_0^s \to J/\psi\phi)}{dp_T} \right)_{ 7 \text{ GeV} \leq p_T < 50 \text{ GeV}} = (28.66 \pm 1.716) \frac{\mu\text{b}}{\text{GeV}}$$

\cleardoublepage

\chapter*{Resumen}
\chaptermark{Resumen}
\label{chap:Resumen}
\addcontentsline{toc}{chapter}{Resumen}
%Insert your abstract below the line
%-------------------------------------------
En esta tesis se ha medido el espectro del mesón $B^0_s$ a través del canal de desintegración $B^0_s \to J/\psi \phi \to \mu^{+}\mu^{-} K^{+}K^{-}$ utilizando los datos de colisiones p-Pb recogidos durante 2016 por el detector CMS en el LHC con energía de centro de masa $\sqrt{s_{NN}} = 8,16$ TeV y luminosidad total integrada 179.1 nb$^{-1}$. También se han considerado muestras simuladas de Monte Carlo con condiciones similares a las de la colisión. La medición se ha realizado mediante una función de densidad de probabilidad para los eventos de ruido y señal y la implementación del método de máxima verosimilitud extendida con el uso del paquete \verb|ROOTFIT|. 

Posteriormente, se calculó la sección transversal diferencial con respecto al momento transversal $p_T$, así como las fuentes de incertidumbre sistemática y estadística. Se realizó una comparación con la sección transversal diferencial de FONLL. Los resultados obtenidos son los siguientes:


$$M(B^0_s) = 5367.02 \pm 0.86 MeV $$
$$\left(\frac{d \sigma(B_0^s \to J/\psi\phi)}{dp_T} \right)_{ 7 \text{ GeV} \leq p_T < 50 \text{ GeV}} = (28.66 \pm 1.716) \frac{\mu\text{b}}{\text{GeV}}$$

