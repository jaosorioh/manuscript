\chapter[\leavevmode\newline Simulation and $B^0_s$ Reconstruction]{Simulation and $B^0_s$ Reconstruction}
\chaptermark{Simulation and $B^0_s$ Reconstruction}
\label{chap:Chapter_3}
\section{Data and Monte Carlo samples}
The official dataset used for analysis was the \verb|/PADoubleMuon/PARun2016C-PromptReco-v1/AO| dataset and it was collected with the CMS experiment during the 2016 proton-nucleus (p-Pb) collisions at $\sqrt{s_{NN}} = 8.16$ TeV, with a total integrated luminosity of $\mathcal{L} = 179.7$ nb$^{-1}$. Monte Carlo (MC) samples are generated prior to using this dataset. These samples can be used to estimate the detector efficiency, the performance of measurement methods described in Chapter 4, and choosing the parameters for reconstruction  of the $B^0_s \to J/\psi \ \phi(1020)$ channel with the real dataset. In order to generate the MC examples, the general-purpose generator \verb|PYTHIA 8| \cite{sjostrand2015introduction} is used to simulate the particle production and hadronization process. The decay of the b hadrons is modeled with the \verb|EVTGEN| package \cite{lange2001evtgen}. The \verb|EVTGEN| module \verb|PHOTOS| \cite{davidson2016photos} calculates the final state radiation (FSR). Lastly, the response of the CMS detector is simulated by the software \verb|GEANT4| \cite{agostinelli2003geant4}. This software is set up with the same triggers and reconstruction algorithms used for the dataset of the collision. %The total number MC samples generated was 

\section{Triggers}
As explained in \ref{subsection:trig_sys} a trigger needs to be implemented in order to select the events of real interest. For the current work, a high level filter labeled as \verb|HLT_PAL1DoubleMuOpen_v1| is used. This is a rather loose trigger that softens the preselection criteria of the two muon candidates to maximize detection efficiency. It should be noted that a trigger prescale is sometimes used, this refers to the selection of only one event from a set of events that bypass the trigger requirements while disregarding the rest \cite{dorigo_2014}. However, no prescaling was used during the 2016 p-Pb collisions, resulting in a higher statistical precision.

\section{Reconstruction and selection criteria}%triggers
To reconstruct the $B^0_s$ meson via the chosen channel, the $J/\psi$ and $\phi (1020)$ mesons must first be reconstructed. For the $J/\psi$ meson, two muon candidates are required, and two kaons are required for the $\phi (1020)$ meson. The specific selection criteria for such candidates will be described in the following sub-sections.
\subsection{J/$\psi$ reconstruction}

Only pairs of muons with opposite charge sign are considered ($\mu^{+}\mu^{-}$). Both muons are required to classify as Soft Muons. They must originate from a common vertex, with a  $\chi^2$ vertex probability, $J_{pro} > 1\%$. It is also necessary for both muons to have $|\eta| < 2.4$ and the following $p_T$:

\[ \begin{cases} 
	p_T > 3.3 \ \text{GeV} & |\eta| < 1.1 \\
	p_T > \left(5.5 - 2.0 \times |\eta|\right) \ \text{GeV} & 1.1 \leq |\eta| < 2.1 \\
	p_T > 1.5 \ \text{GeV} & 2.1 \leq |\eta| < 2.4 
\end{cases}
\] 

As a final condition for selecting a muon pair, the dimuon invariant mass must be in the range $[2.9, 3.3]$ GeV.
\subsection{$\phi(1020)$ reconstruction}
After selecting the appropriate muon candidate tracks, and because the CMS detector cannot correctly identify the tracks, it is assumed that the remaining tracks are kaon tracks, and thus all of them are initially considered for the $\phi(1020)$ reconstruction. However, in order for two pairs of kaons to be valid candidates, they must have opposite charge ($K^{+}K^{-}$), $|\eta| < 2.4$ and have high purity. The invariant mass of the kaon-pair must be between $0.010$ GeV around the reported mass for $\phi(1020)$ in the PDG of $1.01946$ GeV.
\subsection{$B_s^0$ reconstruction}
Once the $J/\psi$ and $\phi(1020)$ candidates are properly reconstructed, they are required to have a common vertex, which corresponds to the $B_s^0$ vertex. The invariant mass of the $J/\psi$ - $\phi$ pair must be in the range $[5.240, 5.490]$ GeV, also the $p_T$ for the $B_0^s$ candidate must be in the range $[7, 50]$ GeV. 

Due to the presence of high combinational background, it is important to find a set of additional conditions in the selection of the $B^0_s$ candidates, among the many that could be established, that reduces the number of background events while increasing the number of signal events. This can be accomplished by maximizing the signal's statistical significance, which is defined as:

\begin{equation}
	\label{eq:sig}
	S = \frac{N_{\text{s}}}{\sqrt{N_{\text{s}} + N_{\text{b}}}}
\end{equation}

With respect to a set of parameters. The parameters used in this case were the decay-length significance, $c\tau / \sigma_{c\tau}$, $p_T$ of kaons and $\chi^2$ vertex probability of $B^0_s$, $B_{pro}$ and the following conditions where imposed:

\begin{itemize}
	\item  $(p_T)_{K_i} > a$, with $a \in [0.5, 1.2]$ GeV.
	\item $B_{pro} > b$ with $b \in [1, 10] \%$ 
	\item $c\tau / \sigma_{c\tau} < c$ with $c \in [1, 7]$
\end{itemize}

$a$, $b$, and $c$ are the values for which $S$ is maximum, and they were calculated by combining various alternative values for $a$, $b$, and $c$. A fit of the invariant mass is done for each combination, and $S$ is determined. The different values are compared, and $max(S)$ is found. The results for these values were $a = 0.8$ GeV, $b = 3\%$ and $c = 6.0$.

A summary of the selection criteria for the reconstruction of the three objects is presented in table \ref{table:sel_criteria}.

\setlength{\tabcolsep}{0.5em} % for the horizontal padding
{\renewcommand{\arraystretch}{1.4}% for the vertical padding
	\begin{table}[!htp]
		\begin{center}
			\begin{tabular}{cc}
				\hline                                                         \multirow{2}{*}{\textbf{Reconstruction}}    & \multirow{2}{*}{\textbf{Selection Criteria}}                                                                                                                                                                            \\
				&                                                                                                                                                                                                                         \\ \hline
				\multirow{5}{*}{$J/\psi \to \mu^{+}\mu{-}$} & Soft-Muon                                                                                                                                                                                                               \\ \cline{2-2} 
				& $|\eta|_{\mu_i} < 2.4$                                                                                                                                                                                                          \\ \cline{2-2} 
				& \begin{tabular}[c]{@{}c@{}}$(p_T)_{\mu_i} > 3.3 \text{GeV if } |\eta|_{\mu_i} < 1.1$\\ $(p_T)_{\mu_i} > \left(5.5 - 2.0 \times |\eta|_{\mu_i}\right) \text{GeV if }  1.1 \leq |\eta|_{\mu_i} < 2.1$\\ $(p_T)_{\mu_i} > 1.5 \text{GeV if } 2.1 \leq |\eta|_{\mu_i} < 2.4$\end{tabular} \\ \cline{2-2} 
				& $M(\mu^{+}\mu^{-}) \in [2.9, 3.3] GeV$                                                                                                                                                                                  \\ \cline{2-2} 
				& $\chi^2$ vertex probability $ > 1 \%$                                                                                                                                                                                   \\ \hline
				\multirow{4}{*}{$\phi(1020) \to K^{+}K{-}$} & High-Purity Track                                                                                                                                                                                                       \\ \cline{2-2} 
				& $|\eta|_{K_i} < 2.4$                                                                                                                                                                                                          \\ \cline{2-2} 
				& $(p_T)_{K_i} > 0.8$ GeV                                                                                                                                                                                                         \\ \cline{2-2} 
				& $|M(K^{+}K^{-}) - M_{PDG}(\phi)| <  0.010$ GeV                                                                                                                                                                          \\ \hline
				\multirow{4}{*}{$B^0_s \to J/\psi \phi$}    & $p_T \in [7.0, 50.0]$ GeV                                                                                                                                                                                               \\ \cline{2-2} 
				& $M(J/\psi \phi) \in [5.240, 5.490] $ GeV                                                                                                                                                                                \\ \cline{2-2} 
				& $c\tau / \sigma_{c\tau}< 6.0 $                                                                                                                                                                                          \\ \cline{2-2} 
				& $\chi^2$ vertex probability $ > 3 \%$                                                                                                                                                                         \\ \hline
			\end{tabular}
			
		\end{center}
	\caption{Criteria for the selection and reconstruction of the $B^0_s \to J/\psi \phi$ decay using the $J/\psi \phi \to \mu^{+}\mu^{-}$ and $\phi(1020) \to K^{+}K^{-}$ decays.}
	\label{table:sel_criteria}
	\end{table}

\begin{section}{CMSSW Framework}
The simulation and event reconstruction described above were all carried out using \verb|CMSSW|, a \verb|C++|-based framework provided by CMS. \verb|Python| is widely used as well. \verb|CMSSW| was designed to make it easier to analyze and process data from detectors \cite{di2020measurement, twiki2013}. To do so, it employs an event data model, in which for each event that passes the trigger selection, a \verb|C++| class known as \verb|Event| is generated \cite{fedi2016studies, muhammad2021measurement}. This class contains all the necessary information about the event. After processing the data with  \verb|CMSSW|, a structured and user-friendly file format known as  \verb|ROOT| is generated, which can then be used for analysis by physicists, such as the one described in the following chapter \cite{di2020measurement}.
\end{section}
