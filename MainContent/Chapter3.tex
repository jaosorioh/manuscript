\chapter[\leavevmode\newline Simulation, Selection and Reconstruction]{Simulation, Selection and Reconstruction}
\chaptermark{Simulation, Selection and Reconstruction}
\label{chap:Chapter_3}
\section{Data and Monte Carlo Samples}
The dataset used for the present analysis corresponds to
an integrated luminosity of 179.7., and it was collected with the CMS experiment during the 2016 proton-nucleus collisions at $\sqrt{s_{NN}} = 8.16$ TeV. On the other hand, in particle physics, it is common practice to use Monte Carlo (MC) simulated samples to estimate detector efficiency and optimal parameters for particle reconstruction. In the present manuscript, a simulation chain consisting of three major steps was used to generate the MC samples. First, the general-purpose generator Pythia is used to simulate particle production. The decay process of the generated hadrons $B^0_s$ is then carried out using EVTGEN. Finally, the particles are detected using the GEANT software, which is a reliable CMS detector simulator. The total number MC samples generated was 
\section{Selection Criteria}%triggers
Two muons with different charges are chosen.
\section{J/$\psi$ Reconstruction}
\section{$\phi(1020)$ Reconstruction}
\section{$B_s^0$ Reconstruction}
\lipsum[3]