\chapter[\leavevmode\newline Conclusions]{Conclusions}
\chaptermark{Conclusions}
\label{chap:Chapter_6}
In summary, the first measurement of the $B^0_s$ differential cross-section $\frac{d\sigma}{dp_T}$ in pPb collisions at $\sqrt{s_{NN}} = 8.16$ TeV is presented. The data was collected by CMS in the LHC during 2016 and corresponds to a total integrated luminosity of $\mathcal L = 179.1$ nb$^{-1}$. The decay chain $B^0_s \to J/\psi \phi \to \mu^{+}\mu^{-} K^{+}K^{-}$ is used to reconstruct the $B^0_s$ mass, as well to estimate the total efficiency of reconstruction. The results are compared with FONLL and are found to lie within the confidence intervals of the uncertainty. The following is the value obtained:
\begin{equation}\left(\frac{d \sigma(B_0^s \to J/\psi\phi)}{dp_T} \right)_{ 7 \text{ GeV} \leq p_T < 50 \text{ GeV}} = (24.834 \pm 1.435(stat) \pm 5.680(sys)) \ \mu\text{b}/\text{GeV}
\end{equation}

This result can be improved by reweighing the MC data, for example. Furthermore, the MC data used in this analysis was created exclusively for this thesis. A more rigorous study would require the use of CMS's official MC data for the 8.16 TeV p-Pb collision. This thesis can also be used as a foundation to investigate energy loss with flavor in heavy-ion collisions in more advanced works, such as a master's or Ph.D thesis.