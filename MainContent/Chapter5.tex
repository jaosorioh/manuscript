\chapter[\leavevmode\newline Differential cross-section of $B^0_s \to J\psi \phi$]{Differential cross-section of $B^0_s \to J\psi \phi$}
\chaptermark{Differential cross-section of $B^0_s \to J\psi \phi$}
\label{chap:Chapter_5}

\section{Total efficiency}

The acceptance $\alpha$ is defined as the ratio of the number of $B^0_s$ mesons that meet the filter criteria for the kaons and muons candidates, table \ref{table:sel_criteria}, to the total number of $B^0_s$ mesons created in the MC simulation,

\begin{equation}
\alpha(p_T, |y|) = \frac{N(B^0_s; \ p_T, |y| ; \ \mathrm{filter \ criteria})}{N(B^0_s; \ p_T, |y| )}
\end{equation}

This value depends on the desired region for both $p_T$ and $|y|$. Because this manuscript takes into account several bins for these variables, a separate acceptance is obtained for each bin, as well as a total acceptance for the entire region. 

The reconstruction efficiency, $\epsilon$, is defined as the ratio of the number of $B^0_s$ mesons that meet all of the selection criteria from table \ref{table:sel_criteria} to the number of $B^0_s$ mesons that meet the filter criteria for kaons and muons candidates (the numerator of acceptance),

\begin{equation}
	\epsilon(p_T, |y|) = \frac{N(B^0_s; \ p_T, |y| ; \ \mathrm{full \ selection \ criteria})}{N(B^0_s; \ p_T, |y| ; \ \mathrm{filter \ criteria})}
\end{equation}

The overall efficiency $\alpha \cdot \epsilon$, is the fraction of the generated $B^0_s$ mesons that actually correspond to the required candidates for investigation, that is, the mesons that decay through the channel studied in this work,

\begin{equation}
	\alpha \cdot \epsilon = \frac{N(B^0_s; \ p_T, |y| ; \ \mathrm{full \ selection \ criteria})}{N(B^0_s; \ p_T, |y|)}
\end{equation}

Fig. and fig. show the acceptance, efficiency and total efficiency for the $p_T$ and $|y|$ bins respectively. 
\section{Differential cross-section}

In particle physics, the term cross-section is used to express the probability of a given event occurring \cite{thomson2013modern}. More specifically, the probability that two colliding particles would interact and generate a certain event, such as the production of a specific particle \cite{pivarski2013}. The cross-section will be used in this thesis to refer to the probability of producing a $B^0_s$ meson during p-Pb collisions that decays in the manner described in \ref{subsec:channel}.

It is sometimes desirable to investigate how a specific kinematic quantity is distributed in respect to the cross-section \cite{thomson2013modern}. This could be valuable in gaining a better understanding of the interaction. For this, the differential cross-section might be used. The differential cross-section with respecto to the transverse momentum, $p_T$, is calculated as follows \cite{abe1995measurement}: 

\begin{equation}
	\label{eq:cs}
\frac{d \sigma(B_0^s \to J/\psi\phi)}{dp_T} = \frac{N_s}{2 \Delta p_T \cdot \alpha \cdot \epsilon \cdot BR \cdot \mathcal{L}}
\end{equation}

where $\Delta p_T$ is the width of the $p_T$ region, $\alpha \cdot \epsilon$ is the total efficiency, BR is the branching ratio of the total decay chain, eq. \ref{eq:br}, and $\mathcal{L} = 179.7$ nb$^{-1}$  is the total integrated luminosity. Due of the quick transitions between particle and antiparticle indicated in \ref{sec:b0s}, both $B^0_s$ and $\bar{B^0_s}$ are formed in the collision, but only $B^0_s$ is considered, for this reason, the factor $\frac{1}{2}$ is included in the equation above.

\section{Uncertainty in the differential cross-section}
The uncertainties in the measurement of a quantity during a physics experiment can be classified by two types. The first type are the statistical uncertainties, which are related to the fact that multiple measurements of the same quantity can yield different results. Therefore, the value of such quantity is not precise, but fluctuates within a range. The measurement of this range is the statistical uncertainty \cite{sinervo2003definition}. The statistical uncertainties are calculated by \verb|ROOTFIT|, and in the case of the ML method, these are determined by the covariance matrix $\mathrm{var}[\vec{\Theta}]$ \cite{vsirca2016probability}:

\begin{equation}
	\mathrm{var}[\vec{\Theta}] = 
	\left(-E\left[ \frac{\partial^2 l(x | \vec{\Theta}) }{\partial \vec{\Theta} ^2}\right]_{\Theta = \hat{\Theta}}\right)^{-1}
\end{equation}

with $E[X]$ the expected value of $X$. The diagonals are the variance of the parameters $\vec{\Theta}$ and the uncertainty is the square root of the variance, 

\begin{equation}
	\delta \Theta_i = \sqrt{\mathrm{var}[\Theta_i]}
\end{equation}

On the other hand, given a set of $N$ random variables $\vec{X}$ and a respective covariance matrix $\mathrm{var}[\vec{X}]$, if a function $f = f(\vec{X})$ depends on these variables, then the variance associated to $f$ is calculated by \cite{vsirca2016probability}:

\begin{equation}
	\mathrm{var}[f(\vec{X})] = \sum_{i=1}^N \sum_{j=1}^N \left(\frac{\partial f}{\partial X_i} \frac{\partial f}{\partial X_j} \right)_{X = \mu} \mathrm{var}[\vec{X}]_{i,j}
\end{equation}

with the partial derivatives evaluated at the mean value of $X$, $\mu$. When there is no correlation between variables, the previous equation reduces to:

\begin{equation}
	\label{totaluncertainty}
	\mathrm{var}[f(\vec{X})] = \sum_{i=1}^N  \left(\frac{\partial f}{\partial X_i}\right)^2_{X = \mu} \mathrm{var}[X_i]
\end{equation}

Revise \cite{vsirca2016probability} for more information on expected values, variance, and their relation to uncertainties. Alternatively, any statistical book on the subject would suffice.

The second type are systematic uncertainties. They are related to the nature of the instrument used, the specific model chosen for the data, the assumptions made about the experiment beforehand, among other factors. It should be noted that by taking more measurements of the quantity, the value of the statistical uncertainty can be minimized. For systematic uncertainties, however, this is not the case \cite{sinervo2003definition}. The uncertainties from different measurements are correlated, which means that they are not independent of one another. After a thorough examination and testing of the potential sources of uncertainty, systematic uncertainties can be calculated and possibly reduced. The total systematic uncertainty is calculated as:

\begin{equation}
	\delta_{sys} = \sqrt{\sum_{i}^{N} \delta_{{sys}_{i}}^2}
\end{equation}

where $\delta_{{sys}_i}$ represents each one of the individual systematic uncertainties.
%\begin{equation}
%	\delta_T = \sqrt{\delta_{stat}^2 + \delta_{sys}^2}
%\end{equation}

In the estimation of the uncertainty of the differential cross-section, several sources of statistical and systematic uncertainties are to be considered. By using eq. \ref{totaluncertainty} in eq. \ref{eq:cs}, the total statistical uncertainty can be obtained:

\begin{equation}
	\delta \left(\frac{d\sigma(B_0^s)}{dp_T} \right)_{stat} 
 =\frac{d \sigma(B_0^s)}{dp_T}\left| \frac{\delta N_s}{N_s}\right|
 \end{equation}

In this situation, only the number of signal events $N_s$ is taken into account. An explanation on how to determine the sources of systematic uncertainties is provided below. 

\subsection{Fit model}

The influence of the signal and background PDFs is analyzed independently to establish the systematic uncertainty deriving from the chosen invariant mass PDF. To accomplish this, one of the PDFs is maintained unmodified, while the other is replaced with a PDF that is also suitable for the data.

As described in section \ref{mlmethod}, the signal PDF has fixed parameters, therefore, a PDF with free parameters is used. Once the fit is completed, a second value for the number of signal events is obtained $N_s'$, and the difference between the original value $N_s$ and $N_s'$ is considered the source of systematic uncertainty in the signal model, $\sigma_{sig}$.

The systematic uncertainties from the background are calculated in a similar fashion. The background PDF is replaced by a PDF consisting of a linear combination of Chebyshev polynomials of the first kind:

\begin{equation}
	B_{PDF}^{'}(M_i) = \sum_{i=0}^{N} c_i T_i(M_i) 
\end{equation}

with \cite{mason2002chebyshev} $T_i(M_i) = \cos(iM_i)$ and $c_i$ the coefficients. The \verb|ROOTFIT| library assumes that for $T_0 = 1$, the coefficient is simply $c_i = 1$ \cite{chebyshev}. Using this model, a second value $N_b'$ for the number of background events is obtained and the systematic uncertainty from the background, $\sigma_{bkg}$, is derived from difference between $N_b$ and $N_b'$.
\subsection{Efficiency}

\subsubsection{MC size}
The uncertainties of the ratios defined for the acceptance and efficiency, $\delta \alpha$ and $\delta \epsilon$ are statistical. However, due to the finite size of the number of Monte Carlo samples used for reconstruction, they turn into systematic uncertainties when considering the differential cross-section. For practical purposes, the uncertainty of the total efficiency $\delta(\alpha \cdot \epsilon)$ is used instead of the individual uncertainties.
\subsubsection{Track}

The uncertainty in the efficiency of reconstruction of the tracks ($K^{+}K^{-}$) is considered a systematic uncertainty. In this paper, the value reported by \cite{cms2018tracking} will be used as the track efficiency uncertainty: $2.4\%$.