\chapter{\leavevmode\newline The Large Hadron Collider}
\label{chap:chapter_2}

The Large Hadron Collider (LHC) is the largest particle collider in the world. It is located inside a 26.7 km long underground tunnel in the facilities of the European Organization for Nuclear Research (CERN), near Ginebra. This tunnel was used in the past for the Large Positron-Electron Collider (LEP) before it was shut down in 2000.

The initial goal of the LHC was to detect the Higgs Boson. For this purpose, it began operations in 2008, but due to a malfunction incident, it was shut down until the end of 2009. Finally, the Higgs boson was detected in 2012. On top of that, many other predictions of the SM have been confirmed using the LHC experimental infrastructure, and it has also been possible to study the phenomena the SM cannot explain, as described in the previous chapter.

The LHC is able to produce proton-proton (p+p) collisions at energies up to 7 TeV and lead ions (Pb-Pb) collisions at 2.76 TeV. p-Pb collisions are also possible at 5.02 TeV \cite{vovchenko2019canonical} and are of special interest in this manuscript. For such collisions, two beams of the target particles are generated using the Proton Synchrotron (PS) and the Super Proton Synchrotron (SPS) and then accelerated in opposite directions at energies up to 460 GeV.

There are four points where collisions occur, and there is a detector associated with each point, as shown in fig \ref{fig:LHC}. Two of these detectors are for general purposes: A Toroidal LHC Apparatus (ATLAS) and Compact Muon Solenoid (CMS). The other two are the LHC beauty detector (LHCb) used to study heavy-flavor physics and indirect CP violations in b-mesons, and A Large Ion Collider Experiment (ALICE) used for heavy-ion collisions.

The results and analysis presented in later chapters are based on CMS. Therefore, a detailed description of this detector will be given in the next section.

\begin{figure}[htp!]
	\centering
	\includegraphics[scale=0.3]{MainContent/Figs/LHC.png}
	\caption{LHC experimental chain. The yellow dots represent the four main experiments. Retrieved from }
	\label{fig:LHC}
\end{figure}

\section{The CMS detector}

CMS is a general purpose detector used to reconstruct the decay products in proton and heavy-ion collisions at high energies. It can detect nearly any particle, especially muons, with high precision. CMS is located in a cavern about 100 m underground near Cessy, France. It has a cylindrical geometry, with a full length of 21.5 m and a diameter of 15 m. With a total weight of 12500 t, it is the heaviest detector in the LHC. CMS is operated by a large collaboration of members worldwide, consisting of over 4000 particle physicists, engineers, computer scientists, technicians, and students from around 200 institutes and universities from more than 40 countries \cite{cms_collab}. The University of Antioquia is one of the collaborating universities through the Phenomenology and Fundamental Interactions Group (GFIF).

The physical structure of the detector consists of a superconducting solenoid able to produce a internal, uniform 4T magnetic field. Inside the solenoid, there is a tracking system, also known as tracker, surrounded by a calorimetry system consisting of the Electromagnetic Calorimeter (ECAL) and the Hadron Calorimeter (HCAL). Outside the solenoid, there is a muon detector chamber. Fig \ref{fig:CMS_structure}  shows a schematic view of the detector.

\begin{figure}[htp!]
	\centering
	\includegraphics[scale=0.2]{MainContent/Figs/cms_structure.png}
	\caption{CMS internal structure with its main sub-systems. Retrieved from }
	\label{fig:CMS_structure}
\end{figure}


Before delving deeper into the detector internal components, the coordinate system will be introduced in the following subsection.

\subsection{The coordinate system}
CMS uses a right-handed Cartesian coordinate system, with the origin at the center of the detector. The $x$-axis points towards the center of LHC, the $y$-axis points upwards, and the $z$-axis points in the counterclockwise direction of the beams. The $x-y$ plane is called the transverse plane and is perpendicular to the beam axis. In this plane, two quantities can be defined: the azimuthal angle $\phi$ with respect to the x-axis and the particle transverse momentum, $p_T = \sqrt{p_x^2 + p_y^2}$. The polar angle $\theta$, on the other hand, is defined in the $z-y$ plane and measured from the $z$-axis. Fig illustrates the coordinate system.

In this coordinate system, one can define two relativistic invariant quantities, the particle rapidity,

\begin{equation}
y = \frac{1}{2}\ln\left(\frac{E+p_z}{E-p_z}\right)
\end{equation}

and the pseudo-rapidity,

\begin{equation}
\eta = -\frac{1}{2}\ln\left(\frac{\theta}{2}\right)
\end{equation}

It is important to mention that the quantities $p_T$ and $\eta$ are of special interest in the results presented in Chapter 4.

\subsection{Superconducting Solenoid}
Located at its center, the superconducting solenoid is the key part of the CMS detector. The solenoid is made of four layers of Nb-Ti coils and has a length of 12.5 m and a diameter of 6 m. It has an working temperature of 4.5K and it can produce a constant magnetic field inside of 4T, however, it is operated at 3.8T. A return yoke made of iron reduces the magnetic field outside the solenoid to about 2T. The magnet stores a total energy of 2.6 GJ. The magnetic field is used to bend the trajectory of the charged particles after the collision, thus allowing to measure their transversal momentum, $p_T$, and charge sign. The stronger the magnetic field, the larger the bending, and, therefore, more precise measurements can be made.

\subsection{The tracker}
\subsection{Calorimetry system}
\subsection{Muon detector}